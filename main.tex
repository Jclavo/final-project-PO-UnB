%%%%%%%%%%%%%%%%%%%%%%%%%%%%%%%%%%%%%%%%%%%%%%%%%%%%%%%%%%%%%%%%%%%%%%
% How to use writeLaTeX: 
%
% You edit the source code here on the left, and the preview on the
% right shows you the result within a few seconds.
%
% Bookmark this page and share the URL with your co-authors. They can
% edit at the same time!
%
% You can upload figures, bibliographies, custom classes and
% styles using the files menu.
%
%%%%%%%%%%%%%%%%%%%%%%%%%%%%%%%%%%%%%%%%%%%%%%%%%%%%%%%%%%%%%%%%%%%%%%

\documentclass[12pt]{article}

\usepackage{PO-final-project}

\usepackage{graphicx,url}

%\usepackage[brazil]{babel}   
\usepackage[utf8]{inputenc}  
\usepackage{attachfile}
     
\sloppy

\title{Designação de tarefas para desenvolvedores}

\author{José Clavo Tafur\inst{1} }


\address{PPGI -- Universidade de Brasilia
  (UnB)\\
  Brasilia -- DF -- Brazil
}

\begin{document} 

\maketitle

\begin{resumo} 
Este projeto descreve ...
%o uso de três técnicas diferentes de métodos quantitativos para comparar, analisar e gerar modelos de predição, tendo como alvo dos algoritmos diferentes que realizam uma análise intraprocedural optimizando programas na linguagem Java.
%Os resultados mostram que a eficiência de executar estas técnicas é um passo a passo completo de como executá-las. 
\end{resumo}


\section{O problema}\label{section:problem}

Numa empresa de TI, o Project Manager (PM) precisa designar a tarefas ao seu equipe de desenvolvedores para desta forma poder entregar os projetos no tempo do deadline.

O processo dentro da empresa tem algumas restrições, por exemplo: que cada desenvolvedor demora um tempo em fazer uma tarefa e cada tem uma quantidade de horas extras. Além disso, a empresa tem um limite nessas horas, o qual não pode ser sobrepassado pelos desenvolvedores.

Portanto, um algoritmo para designar tarefas deve ser implementado tendo em conta as restrições já indicadas.

\section{Motivação}

A pesar dos processos para o desenvolvimento de software são gerenciados usando modernas metodologia, um dos pontos fracos e complicados é a designação de tarefas para os desenvolvedores de uma forma que a quantidade de horas utilizadas seja reduzida, o qual significaria economizar dinheiro e recursos. De tal maneira, usando métodos de ``Linear assigment'' se tenta alcançar este objetivo. 


\section{Modelo de otimização}\label{section:motivation}

Para uma melhor entendimento do problema relatado na (Seção~\ref{section:problem}), a (Figura~\ref{figure:workflow}) mostra o fluxo na empresa e como são as designações.

\begin{figure}[!ht]
	\begin{center}
		\includegraphics[width=1\textwidth]{images/workflow}
	\end{center}
	\begin{center}
		\vskip -0.5cm
		\caption{\label{figure:workflow}
			\small{Fluxo do problema}}
		{\small{Fuente: Elaboración propia.}}
	\end{center}
\end{figure}

Um passo a passo do modelado será mostrado a continuação:

\subsection{Parametrizações}

Os dados de entrada serão:

\begin{center}
	$D = $ Número de desenvolvedores
\end{center}
\begin{center}
	$T = $ Número de tarefas
\end{center}
\begin{center}
	$Aij = $ Matriz do tempo
\end{center}
\begin{center}
	$Oj = $ Vetor do horas extras
\end{center}
\begin{center}
	$MAX_O = $ Límite do horas extras
\end{center}

\subsection{Variável de decisão}

Uma solução válida para o problema seria uma matriz de $(0, 1)$, com o $1$ indicando a designação de um desenvolvedor a uma tarefa e o $0$ indicando o contrario (Figura~\ref{figure:decision_vars}). Aquela matriz será representada por $Xij$ chamada de ``Matriz de designação''.

\begin{figure}[!ht]
	\begin{center}
		\includegraphics[width=1\textwidth]{images/decision_vars}
	\end{center}
	\begin{center}
		\vskip -0.5cm
		\caption{\label{figure:decision_vars}
			\small{Variável de decisão}}
		{\small{Fuente: Elaboración propia.}}
	\end{center}
\end{figure}


\subsection{Restrições}

De acordo com o problema apresentado em (Seção~\ref{section:problem}), as restrições são: 

\begin{enumerate} 
	\item Cada desenvolvedor pode ter ou não tarefas designadas.
	\item Cada tarefa é designada pra um o mais desenvolvedores.
	\item Os desenvolvedores tem um limite de horas extras pra fazer.
\end{enumerate} 

\subsubsection{Restrição 1}
Cada desenvolvedor pode ter ou não tarefas designadas (Figura~\ref{figure:constraint_1}).

\begin{center}\Large
${\sum_{w=0}^{W} X_{wt} \geq 0} , \forall t \in T, \forall w \in W$
\end{center}

\begin{figure}[!ht]
	\begin{center}
		\includegraphics[width=0.7\textwidth]{images/constraint_1}
	\end{center}
	\begin{center}
%		\vskip -0.5cm
		\caption{\label{figure:constraint_1}
			\small{Restrição 1}}
		{\small{Fuente: Elaboración propia.}}
	\end{center}
\end{figure}
 

\subsubsection{Restrição 2}

Cada tarefa é designada pra um o mais desenvolvedores (Figura~\ref{figure:constraint_2}).

\begin{center}\Large
	${\sum_{t=0}^{T} X_{tw} \geq 1}, \forall w \in W , \forall t \in T$
\end{center}

\begin{figure}[!ht]
	\begin{center}
		\includegraphics[width=0.7\textwidth]{images/constraint_2}
	\end{center}
	\begin{center}
		%		\vskip -0.5cm
		\caption{\label{figure:constraint_2}
			\small{Restrição 2}}
		{\small{Fuente: Elaboración propia.}}
	\end{center}
\end{figure}

\subsubsection{Restrição 3}

Os desenvolvedores tem um limite de horas extras pra fazer (Figura~\ref{figure:constraint_3}).

\begin{center}\Large
	${\sum_{w=0}^{W} X_{wt} \times O_t \leq M_O} , \forall t \in T, \forall w \in W$
\end{center}

\begin{figure}[!ht]
	\begin{center}
		\includegraphics[width=1\textwidth]{images/constraint_3}
	\end{center}
	\begin{center}
		%		\vskip -0.5cm
		\caption{\label{figure:constraint_3}
			\small{Restrição 3}}
		{\small{Fuente: Elaboración propia.}}
	\end{center}
\end{figure}


\subsection{Função Objetivo}

A função objetivo é a minimização do tempo para finalizar uma tarefa. 

\begin{center}
${Minimizar}$
\end{center}
\begin{center}\Large
	 ${\sum_{w \in W, t \in T} A_{wt} \times X_{wt}} , \forall t \in T, \forall w \in W$
\end{center}

\subsection{Modelagem Final}

\begin{center}
	${Minimizar}$
\end{center}
\begin{center}\Large
	${\sum_{w \in W, t \in T} A_{wt} \times X_{wt}} , \forall t \in T, \forall w \in W$
\end{center}

\begin{center}
	${Sujeito}$ ${a}$
\end{center}
\begin{center}\Large
	${\sum_{w=0}^{W} X_{wt} \geq 0} , \forall t \in T, \forall w \in W$ (1)
\end{center}
\begin{center}\Large
	${\sum_{t=0}^{T} X_{tw} \geq 1}, \forall w \in W , \forall t \in T$ (2)
\end{center}
\begin{center}\Large
	${\sum_{w=0}^{W} X_{wt} \times O_t \leq M_O} , \forall t \in T, \forall w \in W$ (3)
\end{center}

\section{Resultados}

\subsection{Protótipo}

Para desenvolver o protótipo nos enfocamos em 3 principais seções. Na (Figura~\ref{figure:prototype}) é mostrado o fluxo completo.

\subsection{Dados de Entrada}

\begin{itemize}
	\item \textbf{Dados de entrada}
	
		Como input, se usaram 3 data sets de diferentes tamanhos ``small, medium e big''.\\
		
	\item \textbf{Processo}
		
		Para a implementação se uso a ferramenta do Google chamada OR-tools \cite{ortools_site}, usando um algoritmo de atribuição linear ``Linear Assigment'' no linguagem de programação Python.\\
		
	\item \textbf{Dados de saída}
		Como output teremos o numero de variáveis, o valor objetivo, a matriz de designação $X_{ij}$ e um texto indicando o trabalhador a qual tarefa foi designado e quantas horas tomará faze-la e as horas extras.

\begin{figure}[!ht]
	\begin{center}
		\includegraphics[width=1\textwidth]{images/prototype}
	\end{center}
	\begin{center}
		%		\vskip -0.5cm
		\caption{\label{figure:prototype}
			\small{Prototipo}}
		{\small{Fuente: Elaboración propia.}}
	\end{center}
\end{figure}	
		
\end{itemize}

\subsection{Execuções}

\subsubsection{Dataset: Small}

A execucão destos dados com ``5 desenvolvedores'' e ``4 tarefas'' deu como resultado (Figura~\ref{figure:dataset_small_result})

Se gerou um gráfico de comparação entre a media das horas por tarefa e o resultado de horas por cada tarefa, resultando sempre em um tempo menor a media como é mostrado na (Figura~\ref{figure:dataset_small_compare}). Alem disso, foi calculado o porcentagem de horas economizadas por cada tarefa, isso é mostrado na (Figura~\ref{figure:dataset_small_task_percentage}). 

\begin{figure}[!ht]
	\begin{center}
		\includegraphics[width=1\textwidth]{images/dataset_small_result}
	\end{center}
	\begin{center}
		\caption{\label{figure:dataset_small_result}
			\small{Dataset Small: Resultado}}
		{\small{Fuente: Elaboración propia.}}
	\end{center}
\end{figure}

\begin{figure}[!ht]
	\begin{center}
		\includegraphics[width=1\textwidth]{images/dataset_small_compare}
	\end{center}
	\begin{center}
		\caption{\label{figure:dataset_small_compare}
			\small{Dataset Small: Comparação entre media e resultado}}
		{\small{Fuente: Elaboración propia.}}
	\end{center}
\end{figure}

\begin{figure}[!ht]
	\begin{center}
		\includegraphics[width=1\textwidth]{images/dataset_small_task_percentage}
	\end{center}
	\begin{center}
		\caption{\label{figure:dataset_small_task_percentage}
			\small{Dataset Small: Diferença em \% entre a media e o resultado}}
		{\small{Fuente: Elaboración propia.}}
	\end{center}
\end{figure}



\section{Conclusões}








%\section{O problema}
%
%A otimização de programas é o processo de modificar um programa para fazê-lo executar mais eficientemente ou  usar menos recursos \cite{program_analysis}, para realizar este processo no nível de código fonte se usam análises intra e inter procedurais. Nesta pesquisa, nos enfocaremos na comparação de algoritmos que levam a cabo uma análise intraprocedural \cite{soot_book}, de programas java, a qual é executada no âmbito de um procedimento (método no Java).
%
%Esta comparação fará uso de 3 técnicas sobre métodos quantitativos: comparações pareadas, projeto ${2^3}$ e regressão linear \cite{method_quantitative}.
%
%
%\section{Relevância do problema}
%
%Apesar de que os programas estão sendo criados por seniores desenvolvedores e seus processos são gerenciados usando modernas metodologias, os resultados não são os esperados em relação a sua eficiência. O que se procura é que um programa de computador seja otimizado para rodar mais rapidamente ou para torná-lo capaz de operar com menos armazenamento de memória ou outros recursos, ou consumir menos energia.
%
%O resultado deste estudo apresentará quão confiáveis são as técnicas usadas na análise intraprocedural através da execução de casos de teste sendo validados usando métodos quantitativos.
%
%\section{Trabalho relacionado}
%
%A principal inspiração para a criação do Jimple Framework foi o Soot \cite{soot_web}. O qual é um framework para manipular e otimizar aplicativos Android e Java através de linguagem intermediárias. Além disso, este framework desempenha diferentes tipos de análises como: (Call-graph construction, Points-to analysis, Def/use chains,  Intra e inter procedural data-flow analysis, Taint analysis, entre outros).
%
%
%\section{Metodologia de trabalho}
%
%Inicialmente, a metodologia vai envolver uma revisão do código dos dois algoritmos Jimple Framework \cite{jimple_web} e o White Language \cite{while_lang_web}, os quais estão em Github. Em seguida, conduzimos a coleta dos dados de entrada que serão usados nos testes. Posteriormente, se executaram os casos de testes coletando o tempo de execução deles. Logo depois, se compararam os valores usando os métodos quantitativos. Finalmente, os resultados vão ser apresentados num relatório.
%
%\section{Solução}
%
%
%\subsection{Algoritmos}
%
%Os dois algoritmos a comparar executam uma análise intraprocedural. O primeiro faz parte do ``Jimple Framework'' \cite{jimple_web} e foi desenvolvido na linguagem Rascal. Por outro lado, o segundo algoritmo faz parte do ``White Language'' \cite{while_lang_web} e foi desenvolvido na linguagem Scala. 
%
%Para uma melhor compreensão o primeiro algoritmo será chamado de ``Algoritmo A'' e o segundo de ``Algoritmo B''.
%
%
%\subsection{Técnicas de métodos quantitativos}
%
%As técnicas de métodos quantitativos ao usar são: comparações  pareadas, projeto ${2^3}$ e regressão linear.
%
%\subsubsection{Comparações  Pareadas}
%
%Nesta seção, comparamos os desempenho dos dois algoritmos (A e B). Ambos foram testados 8 vezes com a quantidade de tempo mostrado na (Figura~\ref{figure:comparacoes_pareadas_table}) y en forma gráfica en la (Figura~\ref{figure:comparacoes_pareadas_graph}). os dois algoritmos usaram o mesmo padrão para obter os resultados em segundos.
%
%\begin{figure}[!ht]
%	\begin{center}
%		\includegraphics[width=1\textwidth]{images/comparacoes_pareadas_table}
%	\end{center}
%	\begin{center}
%%		\vskip 0.5cm
%		\caption{\label{figure:comparacoes_pareadas_table}
%			\small{Comparações  Pareadas : valores}}
%		{\small{Fuente: Elaboración propia.}}
%	\end{center}
%\end{figure}
%
%
%\begin{figure}[!ht]
%	\begin{center}
%		\includegraphics[width=1\textwidth]{images/comparacoes_pareadas_graph}
%	\end{center}
%	\begin{center}
%		\vskip -0.5cm
%		\caption{\label{figure:comparacoes_pareadas_graph}
%			\small{Comparações  Pareadas : gráfico}}
%		{\small{Fuente: Elaboración propia.}}
%	\end{center}
%\end{figure}
%
%Se compararam com dois valores de intervalos.
%
%\begin{itemize}
%	\item \textbf{intervalo 90 \%}
%		(-0,081 ; 0,331)
%		
%		Neste intervalo, não se pode rejeitar a hipótese já que tem o 0 entre os valores. Portanto, os algoritmos têm desempenho similar.
%		
%	\item \textbf{intervalo 80\%}
%		(1,68 ; 1,99)
%		
%		Neste intervalo, pode-se argumentar que A tem desempenho melhor que B.	 
%		 
%\end{itemize}	
%
%Um completo passo a passo da execução da técnica pode ser baixado aqui \attachfile{pdfs/comparacoes_pareadas_steps.pdf}
%
%\subsubsection{Projeto ${2^3}$}
%
%Nesta seção, analisaremos os 3 fatores que maior impacto em nosso projeto os quais são: tipo do algoritmo, tipo do sistema operativo e quantidade de processadores (Figura~\ref{figure:projeto_2_fatores}).
%
%\begin{figure}[!ht]
%	\begin{center}
%		\includegraphics[width=0.5\textwidth]{images/projeto_2_fatores}
%	\end{center}
%	\begin{center}
%		\caption{\label{figure:projeto_2_fatores}
%			\small{Projeto ${2^3}$: fatores}}
%		{\small{Fuente: Elaboración propia.}}
%	\end{center}
%\end{figure}
%
%O projeto ${2^3}$ e seus valores em segundos são mostrado na (Figura~\ref{figure:projeto_2_valores}).
%
%\begin{figure}[!ht]
%	\begin{center}
%		\includegraphics[width=0.7\textwidth]{images/projeto_2_valores}
%	\end{center}
%	\begin{center}
%		\caption{\label{figure:projeto_2_valores}
%			\small{Projeto ${2^3}$: valores}}
%		{\small{Fuente: Elaboración propia.}}
%	\end{center}
%\end{figure}
%
%A porção de variação explicada por cada fator e suas interações são mostradas na (Figura~\ref{figure:projeto_2_interacoes})
%
%\begin{figure}[!ht]
%	\begin{center}
%		\includegraphics[width=0.7\textwidth]{images/projeto_2_interacoes}
%	\end{center}
%	\begin{center}
%		\caption{\label{figure:projeto_2_interacoes}
%			\small{Projeto ${2^3}$: interações}}
%		{\small{Fuente: Elaboración propia.}}
%	\end{center}
%\end{figure}
%	
%O \textbf{``fator C''} com \textbf{81,27\%} tem a maior variação, portanto a quantidade de processadores são os que geram o maior impacto. 
%
%Um completo passo a passo da execução da técnica pode ser baixado aqui \attachfile{pdfs/projeto_steps.pdf}
%
%\subsubsection{Regressão Linear}
%
%Nesta seção, examinaremos o tempo de execução em segundos do algoritmo A com quantidades diferentes de loops no código (Figura~\ref{figure:regressao_valores}), um loop neste representa um ``while'' ou ``for'', para poder realizar uma predição.
%
%\begin{figure}[!ht]
%	\begin{center}
%		\includegraphics[width=1\textwidth]{images/regressao_valores}
%	\end{center}
%	\begin{center}
%		\caption{\label{figure:regressao_valores}
%			\small{Regressão Linear: valores}}
%		{\small{Fuente: Elaboración propia.}}
%	\end{center}
%\end{figure}
%
%Além disso, se apresenta um gráfico com os parâmetros de estimativa (Figura~\ref{figure:regressao_graph}) para uma melhor visualização.
%
%\begin{figure}[!ht]
%	\begin{center}
%		\includegraphics[width=0.9\textwidth]{images/regressao_graph}
%	\end{center}
%	\begin{center}
%		\vskip -0.5cm
%		\caption{\label{figure:regressao_graph}
%			\small{Regressão Linear: gráfico}}
%		{\small{Fuente: Elaboración propia.}}
%	\end{center}
%\end{figure}
%
%A qualidade da regressão medida pelo coeficiente de determinação ${R^2}$ é \textbf{0,95}, portanto mostra uma alta valor de regressão.
%
%Se efetuou uma predição para um programa com 64 loops, o tempo calculado \textbf{T} é \textbf{19,27}  segundos com \textbf{90\%} de \textbf{IC} é \textbf{(18,66;19,88)}.
%
%Um completo passo a passo da execução da técnica pode ser baixado aqui \attachfile{pdfs/regressao_steps.pdf}
%
%\section{Conclusões}
%
%As técnicas de métodos quantitativos colocadas em prática mostraram o seguinte: A comparação de observações pareadas com um intervalo de 80\% revelou que o algoritmo A é melhor que o B. Além disso, a execução do projeto ${2^3}$ deu como resultado que a quantidade de processadores são os que geram um maior impacto em nosso projeto e finalmente, devido ao resultado das observações pareadas se tomou conta só do algoritmo A com a técnica de regressão linear, a qual gerou um modelo que tem a finalidade de predizer. Este foi testado para um programa com 64 loops e o tempo calculado foi de 19,27 segundos.
%
%Como conclusão pessoal, as técnicas de métodos quantitativos são umas ferramentas interessantes no momento de comparar, analisar e até fazer uma predição. Aqueles devem ser utilizadas (ou se já estão sendo, devem ser exploradas outras técnicas) em nossos projetos.
%
%
%
%
%%\bibliographystyle{sbc}
\bibliographystyle{apalike}
\bibliography{apa-template}


\end{document}
